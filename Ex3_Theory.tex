\documentclass[10pt,a4paper]{article}
\usepackage[utf8]{inputenc}
\usepackage{mathpartir}

\begin{document}
\section*{Ex3 Theory}
\subsection*{2.4}
\begin{mathpar}
\inferrule*[Right=($for$)] 
{	\Gamma\vdash_{e} $ e0 : int$\\
	\Gamma $, (l, i : int) $ \vdash_{e} $ e1 : bool$\\
	\Gamma $, (l, i : int) $ \vdash_{s} $ s1$\\
	\Gamma $, (l, i : int) $ \vdash_{s} $ s2$} 
{\Gamma\vdash $ for (int i=e0; e1; s1) s2$}
\end{mathpar}
\subsection*{2.5}
Derivation (We use $\Gamma$ for (return : int[]), (l, y: int[])) :
\begin{mathpar}
\inferrule*[Right=($var-decl$)] 
{	\inferrule*[Right=($seq$)] {
		\inferrule*[Right=($assign$)] {
		D2 \\
		int[] \prec int[]
		}
		{
		\Gamma \vdash_{s} $ y = new int[2];$
		}\\ \\
		\inferrule*[Right=($seq$)] {
			D3
		}
		{
		\Gamma \vdash_{sl} $ y[1] = 1; return y;$
		}
	}
	{$(return : int[]), (l, y: int[]) $ \vdash_{sl} $ y = new int[2]; y[1] = 1; return 	y;$
	}
} 
{	
	$(return : int[]) $ \vdash_{sl} $ int[] y; y = new int[2]; y[1] = 1; return y;$
}

\end{mathpar}
\\\\D2:
\begin{mathpar}
\inferrule*[Right=($var-use$)] {
		$(\_, y : int) $ \epsilon$ $\Gamma
		}
		{
		\Gamma \vdash_{e} $y : int[]$
		}\\ \\
\inferrule*[Right=($new-array$)] {
		\inferrule*[Right=($int-literal$)] { }
		{
			\Gamma\vdash_{e} $ 2 : int$
		}
}
{
	\Gamma \vdash_{e} $ new int[2] : int[]$
}	
\end{mathpar}
\\\\D3:
\begin{mathpar}
\inferrule*[Right=($assign$)] {
	\inferrule*[Right=($array-lookup$)] 
	{ 	
		\inferrule*[Right=($var-use$)] {$(\_, y : int[]) $\epsilon$ $\Gamma }
		{
		\Gamma \vdash_{e} $ y : int[]$
		}\\ \\ \\
		\inferrule*[Right=($int-literal$)] { }
		{
		\Gamma \vdash_{e} $ 1 : int$
		}
	}
	{
		\Gamma \vdash_{e} $ y[1] : int$
	}\\ \\ \\ \\
	\inferrule*[Right=($int-literal$)] { }
	{
		\Gamma \vdash_{e} $ 1 : int$
	}
}
{
	\Gamma \vdash_{s} $ y[1] = 1;$
}\\\\
\inferrule*[Right=($return$)] {
	\inferrule*[Right=($var-use$)] {
			$(\_, y : int[]) $\epsilon$ $\Gamma
		}
		{
			\Gamma \vdash_{s} $ y : int[]$
		}\\ \\ \\
	$ int[] $ \epsilon $ int[]$\\
	$ (return : int[]) $\epsilon$ $ \Gamma
}
{
	\Gamma \vdash_{s} $ return y;$
}				
\end{mathpar}
\end{document}